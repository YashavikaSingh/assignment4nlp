I designed a combined transformation that applies two complementary modifications to create out-of-distribution (OOD) data while preserving semantic meaning: \textbf{synonym replacement} and \textbf{keyboard typo introduction}. This dual approach ensures that transformed examples maintain the same label as their originals, as a human would assign identical labels to both versions.

\paragraph{Synonym Replacement (20\% probability per word).} For each alphanumeric token, with 20\% probability, the transformation attempts to replace it with a semantically equivalent synonym using WordNet. The implementation: (1) queries WordNet using \texttt{wordnet.synsets(word)} to retrieve all synsets, (2) extracts synonyms from all lemmas across synsets, (3) filters to single-word replacements (excluding multi-word phrases, hyphenated words, and special characters), (4) randomly selects a valid synonym if available, and (5) preserves the original capitalization pattern. If no suitable synonym is found, the transformation proceeds to typo introduction.

\paragraph{Keyboard Typo Introduction (30\% probability per word).} For words not replaced by synonyms, with 30\% probability, the transformation introduces realistic typing errors by replacing a single character with an adjacent key on a QWERTY keyboard. The process: (1) identifies all alphabetic characters and their positions, (2) randomly selects one character position, (3) replaces it with a randomly chosen adjacent key from a pre-defined QWERTY adjacency map (e.g., 'a' $\rightarrow$ 's', 'q', or 'w'), and (4) preserves the original case.

\paragraph{Processing Pipeline.} The complete transformation: (1) tokenizes input text using NLTK's \texttt{word\_tokenize()}, (2) for each token, keeps punctuation unchanged and applies synonym replacement (20\% probability) first, then typo introduction (30\% probability) if synonym replacement doesn't apply, and (3) detokenizes transformed tokens using \texttt{TreebankWordDetokenizer()}.

\paragraph{Why This Transformation is Reasonable.} Both components reflect real-world variations: \textbf{synonym replacement} models natural language variation (e.g., ``movie'' $\rightarrow$ ``film''), while \textbf{keyboard typos} simulate common typing errors due to physical keyboard adjacency (e.g., ``movie'' $\rightarrow$ ``movoe''). Both preserve semantic meaning, ensuring identical labels.

\paragraph{Example.} \textbf{Original:} ``Titanic is the best movie I have ever seen.'' \textbf{Transformed:} ``Titanic is the best film I have ever seen.'' (synonym) or ``Titanic is the best movoe I have ever seen.'' (typo). Both preserve positive sentiment.

